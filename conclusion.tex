\section{New Results}

\begin{itemize}
\item We show a new result for the relativistic normal form factor
      for hydrogenic atoms in equation~\ref{eq:nff-relativistic}.
\item We show that for low Z atoms (non-relativistic limit)
      the relativistic normal form factor reduces to the non
      relativistic normal form factor for hydrogen in equation~\ref{eq:reduction}.
\item We compare and look at the differences between the new relativistic
      form factor and the non relativistic theory~\ref{eq:nff-nonrelativistic}
      and an alternate theory (Hubbell) and graphically
      show the differences in figures~\ref{fig:delta-theory} 
      and~\ref{fig:hubbell-comparison}

\item We show semi-analytic results for the first order relativistic
photoabsorption amplitude for photons polarised in both the $x$ and $y$
directions for the all poles approach (equations~\ref{eq:A1-X}
and~\ref{eq:A1-Y}) 
and for the electric dipole approximation (equation~\ref{eq:A1-Dipole}).

\item We show a new complete analytic result for the first order
photoabsorption amplitude in the forward scattering direction
(equation~\ref{eq:A1-Forward}).

\item We show a result for the second order S-matrix amplitude for the
      all poles (equation~\ref{eq:A2}) and electric dipole 
      approximation (equation ~\ref{eq:A2-Dipole}).

\item We present a result in figure~\ref{fig:compare-poles} for the normal form factor $f''$
      in both the all poles approach and the electric dipole approximation
      over a range of energies.

\item We compare the results obtained for $f''$ near an absorption edge for
      hydrogen with other theories (see figure~\ref{fig:fpp-compare}).

\item We show results for angular the angular distribution of ejected 
      photoelectrons at selected energies (see figure~\ref{fig:angular-plots}).

\end{itemize}

\section{Further Work In This Area}
    The work presented in this thesis can be expanded in a number of different
    ways. Firstly, the integration techniques used in the S-matrix theory can
    be improved to hopefully provide more accurate results.
    Secondly, the first order photoionisation amplitudes can be used
    to calculate a first order photo-electric cross section which 
    can then be used to calculate $f''$, which may then be compared
    to the results obtained with the second order S-matrix theory.
    Thirdly, the cross section components due to bound-bound transitions,
    and other effect such as Delbr\"uck and Compton scattering and pair
    production should be added to the existing photoionisation 
    cross section.

    Longer term research in the area of relativistic atomic form factors
    can include calculations for molecular hydrogen, bound hydrogen, and a study
    of the multiple scattering processes which give rise to XAFS (X-Ray
    Anomalous Fine Structure).
    Additionally, the development form factor calculations for many electron
    atoms ranging from Helium to Uranium making use of numerical 
    relativistic Dirac-Hartree-Fock wave-functions is also of great interest.



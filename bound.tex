
\section{Ground State to an Excited Bound State}
We mentioned in Chapter 5 that ideally the total cross section would have both
amplitudes corresponding to ground-continuum transitions and bound-bound
transitions. The first step that is required is to calculate a photoabsorption
matrix element from the ground state to one of the excited bound
states of hydrogen. 

This chapter provides semi analytic results for photoabsorption matrix elements
from the ground state to the first three excited states of hydrogen.
We will consider the the relativistic absorption operator $\mathcal{A} = \Alpha_x \Exp{k}{r}
= \Alpha_x e^{i\eta(\mb{k},\theta,\phi)r}$ for a photon polarised in the $x$
direction and we define $\eta$ as:
\begin{equation}
    \eta(\mb{k},\theta,\phi) = k_x \sin\phi \cos\phi +
                               k_y \sin\phi \sin\theta +
                               k_z \cos\phi
\end{equation}
In the electric dipole approximation $\mb{k} = 0$ and hence $\eta = 0$. 
The use of $\eta$ allows us to separate the angular and radial integrals
therefore writing the photoabsorption matrix element from the ground
state $\ket{\psi_0}$ to an excited bound state $\ket{\psi_i}$ as:
\begin{equation}
    \bra{\psi_i} X \ket{\psi_0}
    =
    \int (A_{i1}A_{01} + A_{ii}A_{0i}) R^g_{0i} \; d\Omega
    +
    \int (A_{i3}A_{03} + A_{i4}A_{04}) R^h_{0i} \; d\Omega
\end{equation}
where $A_{ij}(\theta,\phi)$ are the angular components of the excited 
state Dirac spinors as defined in Chapter 2.
$R^g_{0i}$ and $R^h_{0i}$ are the large and small component radial 
integrals which are defined in the sections below for each excited state.
Some of the radial integrals are analytic while others have to be solved
numerically. The final integration over all angles $ d\Omega $ has to 
be performed numerically.

All the radial integrals are written in terms of the constants defined in
Chapter 2 and in terms of the integrals $\mathcal{J}$ and $\mathcal{K}$ 
defined below.
\begin{equation}
    \mathcal{J}[a,n] = \int_0^\infty e^{-ar} r^n \; dr
\end{equation}
\begin{equation}
    \mathcal{K}[a,n,A,B,C,D] = \int_0^\infty 
                \left( \frac{A-Br}{C-Dr} \right) e^{-ar} r^{n} \; dr
\end{equation}

\section{Ground State to First Excited State}
\begin{equation}
R^g_{01} = G_0 G'_1 \mathcal{J}[\Half(\sigma_1 + \sigma_2) - i\eta,2\gamma_1] -
           G_0 G''_1 \mathcal{J}[\Half(\sigma_1 + \sigma_2) - i\eta,2\gamma_1 + 1]
\end{equation}

\begin{equation}
\begin{split}
R^h_{01} = H_0 G_0 H_1 
            G'_1 \mathcal{K}[\Half(\sigma_1 + \sigma_2) - i\eta, 2\gamma_1,H'_1,H''_1,H'''_1,H''''_1]
            \\ -
   H_0 G_0 H_1          G''_1 \mathcal{K}[\Half(\sigma_1 + \sigma_2) - i\eta, 2\gamma_1,  1H'_1,H''_1,H'''_1,H''''_1]
\end{split}
\end{equation}

\section{Ground State to Second Excited State}
\begin{equation}
R^g_{02} = G_0 G'_2 \mathcal{J}[\Half(\sigma_1 + \sigma_2) - i\eta,2\gamma_1] -
           G_0 G''_2 \mathcal{J}[\Half(\sigma_1 + \sigma_2) - i\eta,2\gamma_1 + 1]
\end{equation}

\begin{equation}
\begin{split}
R^h_{02} = H_0 G_0 H_2 
            G'_1 \mathcal{K}[\Half(\sigma_1 + \sigma_2) - i\eta, 2\gamma_1,H'_2,H''_2,H'''_2,H''''_2]
            \\ - 
   H_0 G_0 H_2          G''_1 \mathcal{K}[\Half(\sigma_1 + \sigma_2) - i\eta, 2\gamma_1, 1H'_2,H''_2,H'''_2,H''''_2]
\end{split}
\end{equation}

\section{Ground State to Third Excited State}
\begin{equation}
R^g_{03} = G_0 G_3 \mathcal{J}[\Half(\sigma_1 + \sigma_3) - i\eta, \gamma_1 + \gamma_2]
\end{equation}

\begin{equation}
R^h_{03} = H_0 G_0 H_3 G_3 \mathcal{J}[\Half(\sigma_1 + \sigma_3) - i\eta, \gamma_1 + \gamma_2]
\end{equation}


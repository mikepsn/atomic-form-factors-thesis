%==============================================================
% Introduction
\chapter{Introduction}
%==============================================================
\section{Background}
The use of high energy radiation has many wide spread uses. In
particular, radiation in the X ray regime has applications in medicine,
astronomy and in the study of materials. 
One way to better understand how energetic radiation such as X rays
interact with matter is to consider the most fundamental physical
process involved. In this case, it is the quantum mechanical scattering
of X ray photons off a single atom.
The theoretical calculation of the scattering form factor allows the
determination of a number of quantities such as the scattering cross
section and the material's attenuation coefficient. 

\section{Aims}
The main aim of this project is to compute the form factor for photons
in the X ray regime scattering of small atoms, in particular Hydrogen
and Helium. This is calculated using the relativistic wave-functions as
obtained using the Dirac equation for Hydrogen and a Dirac-Hartree-Fock
numerical approach for Helium. The two components of the form factor
that will be computed are the normal component which has an angular
dependence, and the anomalous component which depends on the energy of
the incident photon.

\section{Aims}
The main aim of this project is to compute the form factor for photons
in the X ray regime scattering of small atoms, in particular Hydrogen
and Helium. This is calculated using the relativistic wave-functions as
obtained using the Dirac equation for Hydrogen and a Dirac-Hartree-Fock
numerical approach for Helium. The two components of the form factor
that will be computed are the normal component which has an angular
dependence, and the anomalous component which depends on the energy of
the incident photon.

\section{Atomic Form Factor}
    \subsection{The Physical Model}
    The scattering process considered consists of a single atom in it's
    ground state described by a relativistic four component Dirac spinor
    $\psi$. A photon of momentum $\mb{k}$ and a polarisation $\mb{\hat{e_n}}$
    scatters off the atom. The two processes considered will be:
    \begin{itemize}
        \item Photoionisation of one of the atom's electrons.
        \item Elastic scattering where the momentum of the scattered
              photon $\mb{k'}$ is the same as the incident photon.
    \end{itemize}

    \subsection{Normal Component}
    The normal component of the atomic form factor shows the angular
    dependence of the scattering process. It can be considered as the
    Fourier transform of the electronic charge distribution and is
    defined as:
    \begin{equation} \label{eq:nff-general}
        f_0(q) = \int e^{i \mb{q} \cdot \mb{r} } \rho(\mb{r}) d\mb{r}
    \end{equation}
    where the momentum transfer is given by $\mb{q} = |\mb{k} - \mb{k'}|$ 
    and $\rho(\mb{r})$ is the electron charge density. In non
    relativistic quantum mechanics, this is defined as 
    $\rho(\mb{r}) = \psi^{*}(\mb{r}) \psi(\mb{r})$ whereas in
    relativistic quantum mechanics it is defined as 
    $\rho(\mb{r}) = \psi^{\dagger}(\mb{r}) \psi(\mb{r})$.
    In the case of a spherically symmetric atom, the normal form factor
    is give by:
    \begin{equation} \label{eq:nff-spherical}
        f_0(q) = 4 \pi \int_{0}^{\infty} \rho(r) \frac{\sin (qr)}{q} r \; dr
    \end{equation}

    \subsection{Anomalous Component}
    The anomalous component of the form factor is defined in terms of
    the scattering amplitude. The cases that are considered are the
    excitement of the ground state electron to a higher energy
    unoccupied state and the excitement of the electron to the
    continuum, that is photoionisation.

    If we consider a relativistic four component spinor $\ket{B}$, which
    is solution to a many particle Dirac equation, the second order
    scattering amplitude (S-matrix) is given by: 
    \begin{equation}
        A = - \sum_P
            \left[
                \frac{\bra{M} O_f^* \ket{P} \bra{P} O_i \ket{N} }
                     {E_N - E_P + \hbar \omega_i + i0_+}
                +
                \frac{\bra{M} O_i \ket{P} \bra{P} O_f^* \ket{N} }
                     {E_n - E_p - \hbar \omega_f - i0_+}
            \right]
    \end{equation}
    where $\ket{N}$ is the initial state, $\ket{M}$ is the final state,
    and $\ket{P}$ are intermediate states. We have the following
    definitions:
    \begin{itemize}
        \item $\hbar \omega_i$ is the energy of the incident photon
        \item $\hbar \omega_f$ is the energy of the scattered photon
        \item $E_N, E_M, E_P$ are the energies of the initial, final 
              and the intermediate states respectively.
        \item $0_+$ is a small positive number.             
    \end{itemize}
    The photon absorption operator $O_{i}$ and the photon emission
    operator $O_{f}^{*}$ are defined as:
    \begin{equation}
        O_i = \sum_j q_j \; \mb{\alpha} \cdot \hat{\epsilon}_i \;
              e^{i \mb{k_i} \cdot \mb{r_j} }
    \end{equation}
    \begin{equation}
        O_{f}^{*} = \sum_j q_j \; \mb{\alpha} \cdot \hat{\epsilon}_j \;
            e^{i \mb{k_f} \cdot \mb{r_j} }
    \end{equation}
    with $\mb{\alpha}$ is the Dirac alpha matrix operator and
    $\hat{\epsilon}_k$ is the polarisation of the photon.


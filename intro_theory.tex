
\section{Atomic Form Factor Theory}
The atomic form factor is given by the x-ray scattering amplitude $f$ 
which by convention separated into a number of components. 
It is commonly written as~\cite{Chantler-Book}
\begin{equation}
    f = f_0 + f' + if''
\end{equation}
where $f_0 = f_0(q,Z)$ is known as the normal or coherent scattering amplitude
and is a function of momentum transfer $q$ and atomic number $Z$.
The anomalous scattering component (also known as the anomalous dispersion or
resonant scattering component) has real and imaginary parts given by $f'$ and
$f''$. The anomalous component is a function of the incident photon energy.
In the literature $f'$ and $f''$ are also known as $f_1$ and $f_2$.

\subsection{Normal Component}
The normal component of the atomic form factor is defined as the Fourier
transform of an atom's electronic charge density~\cite{Crasemann}.
The expression for $f_0(q)$ with an atom with electronic charge density
$\rho(\mb{r})$, is given below with the second equation being valid for the case
of a spherically symmetric atom~\cite{Chantler-Book}.
\begin{equation} \label{eq:nff-spherical}
    f_0(q) = \int \rho(\mb{r}) e^{i\mb{q}\cdot\mb{r}} \; d\mb{r}
           = \int_0^\infty \rho(r) \frac{\sin(qr)}{qr} r^2 \; dr
\end{equation}
The momentum transfer is defined as
\begin{equation}
    q = |\mb{k_{final}} - \mb{k_{initial}}| = \frac{4\pi \sin(\theta/2)}{\lambda}
\end{equation}
where $\lambda$ is the wavelength of the incident photon and $\theta$ is the
scattering angle. It is conventional to measure the momentum transfer $q$
in inverse Angstroms ($\InvAngstrom$).
  
\subsection{Anomalous Component}
The imaginary part of the anomalous form factor is related to the total
photoionisation cross section.
\begin{equation} \label{eq:fpp-sigma}
    f''(\omega) = \frac{\omega}{4\pi c r_0} \sigma(\omega)
\end{equation}
The photon energy is given by $\hbar\omega$, $c$ is the speed of light and $r_0$
is the classical electron radius. The photionisation cross section 
$\sigma(\omega)$ may also include cross sections from bound-bound
transitions.

The $f'$ component may be computed from the $f''$ component using a 
dispersion relation.
\begin{equation}
    f'(\omega) = f'(\infty) - \frac{2}{\pi} P
                 \int_0^\infty 
                 \frac{\omega' f''(\omega')}{\omega^2 - {\omega'}^2} \; d\omega'
\end{equation}

\subsection{Theoretical Limitations and Assumptions}
The theoretical calculations of atomic form factors are usually made with a
number of limitations and/or assumptions. These fall into a number of different
categories. Improvements to existing theories attempt to eliminate one or more
of these assumptions or limitations.

\begin{description}
    \item[\it 1. ATOMIC STRUCTURE :] 
    For a simple atom such as hydrogen, the quantum
    mechanical wave-functions of an atom may be either non relativistic (standard
    Schr\"odinger wave-functions), or relativistic (four component Dirac
    spinors). For many-electron atoms there is still a choice between non
    relativistic and relativistic wave-functions but additional choices have
    to be made as how to compute these wave-functions as analytic solutions
    are not available. Methods for computing many electron wave-functions include
    Hartree-Fock (HF), Dirac-Hartree-Fock (DHF), Hartree-Slater (HS), 
    Multi-Configuration-Dirac-Fock
    (MCDF) and all orders methods which include quantum electrodynamic
    corrections~\cite{Sapirstein}. Although issues such as the suitability of the
    independent particle approximation (IPA) and the computational methods used
    to determine atomic structure are important in the areas of atomic physics,
    many body perturbation theory and computational chemistry, they are also of
    relevance to form factor theory due to the reliance on accurate
    computed wave-functions.
    \item[\it 2. ELECTROMAGNETIC FIELD :]
    The incident photon is modeled as either a classical or quantised
    electromagnetic field. The simpler approach of using a classical
    electromagnetic field is more common and approximations to this model
    involve considering only the electric dipole (E1) and/or electric quadrupole
    (E2) approximations. Alternatives includes using a relativistic multipole
    (RMP) or an ``all-poles'' approach can be taken in
    which no approximations to the classical electromagnetic field are made.
    \item[\it 3. ISOLATED ATOM :]
    Most form factor calculations are done for a single isolated atom. However,
    experimentally it is extremely difficult if not impossible to obtain results
    from an isolated atom. As a result, there are inherent limitations in
    theories which consider only the case of an isolated atom. Experimentally,
    effects such as XAFS (X-ray Anomalous Fine Structure) arise from multiple
    scattering processes off multiple atoms.
    \item[\it 4. PERTURBATION THEORY :]
    Models of atom-photon interactions usually treat the electromagnetic field
    model of the photon as a small perturbation, and such there are a number of
    approaches in computing the required matrix elements and relevant scattering
    amplitudes. The main issue involves the order of the perturbation theory 
    (usually first or second order) and the type of perturbation theory used -
    standard (time dependent or time independent), relativistic perturbation
    theory and second order S-matrix theory which is obtained from covariant
    perturbation theory.
    \item[\it 5. ADDITIONAL PROCESSES :]
    A number of processes can occur when a photon interacts with an atom. 
    An atomic form factor calculation usually includes one or more of these
    processes. These processes include:
        \begin{itemize}
            \item Photoionisation
            \item Bound-Bound Transitions
            \item Rayleigh (Coherent) Scattering
            \item Compton (Incoherent) Scattering
            \item Delbr\"uck Scattering
            \item Pair Production
            \item Nuclear Thomson Scattering
        \end{itemize}
    \item[\it 6. NUMERICAL AND COMPUTATIONAL :]
    Form factor calculations require considerable computational work. Therefore
    a number of numerical computation issues need to be considered. These
    includes choices for integration and interpolation methods, numerical
    precision, convergence and computation time.
\end{description}







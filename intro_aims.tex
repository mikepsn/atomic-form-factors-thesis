
\section{Introduction}
The use of high energy radiation has many wide spread uses. In particular,
the understanding of how radiation in the X ray regime interacts with matter 
has applications in areas such as astronomy, medicine and in the study of the
structure of materials such as in the field of crystallography.

In order to be able to interpret results of experiments which make use of X rays
or other high energy photons we must understand how such photons interact with
matter. The most fundamental process that can be considered is the interaction
of a single photon with a single isolated atom. This can include the study of
coherent and incoherent scattering processes, photoionisation, photoabsorption
and pair production.

In particular, the theoretical calculation of atomic form factors allows the
determination of quantities such as cross sections and attenuation coefficients
which are valuable in obtaining information about the atom-photon interaction
process.

The study of atom-photon interactions through the calculation of atomic form
factors covering all elements has been the subject of decades of experimental
and theoretical research. This work includes the synthesis by Henke~\cite{Henke-Experimental}
of experimental and theoretical results, theoretical results using the
relativistic dipole approximation of Cromer and
Libermann~\cite{Cromer-1,Cromer-2,Cromer-3}, theoretical results of Kane et al.
based on S-matrix theory~\cite{Kane-Kissel-Pratt}, recent theoretical
tabulations by Chantler~\cite{Chantler-Tabulation}, the results of
Saloman, Hubbell and Scofield~\cite{Saloman-Hubbell-Scofield} and the results
of Creagh~\cite{Creagh-Hubbell,Creagh-McAuley}.

\section{Project Aims}
Major improvements on current results~\cite{Chantler-Book}
would also involve many years of additional research.
However, a detailed comprehension of the assumptions made in research to date 
may be made by a detailed and critical study of form factor calculations for
hydrogen; in particular looking at the different assumptions and methods used to
compute atomic form factors.
This study of hydrogen allows us to show new results for:
\begin{enumerate}
    \item Analytic forms for the relativistic normal form factor for hydrogenic 
          atoms.
    \item Numerical results for imaginary
          component of the anomalous form factor for hydrogen using a
          relativistic S-matrix approach.
    \item The angular dependence of the photionisation differential cross
          sections at selected energies.
    \item Semi-analytic results for hydrogenic bound-bound transition matrix elements.
\end{enumerate}


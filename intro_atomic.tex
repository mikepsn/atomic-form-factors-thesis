%============================================================================
% ATOMIC STRUCTURE DEFINITIONS AND CONSTANTS
%============================================================================

This section describes the atomic structure information required for the form
factor calculations such as physical constants, definitions, atomic energy
levels and atomic wave functions.

%============================================================================
% CONSTANTS
%============================================================================
\section{Constants and Definitions}
We first need to define the usual symbols which are common in atomic physics.
\begin{center}
\begin{tabular}{|c|l|} \hline
    Symbol                  &   Meaning                     \\ \hline \hline
    $c$                     &   Speed Of Light              \\
    $e$                     &   Electron Charge             \\
    $m$                     &   Electron Mass               \\
    $a_0$                   &   Bohr Radius                 \\ 
    $\alpha$                &   Fine Structure Constant     \\
    $\hbar$                 &   Planck's Constant/$2\pi$    \\
    $r_0$                   &   Classical Electron Radius   \\
    $Z$                     &   Atomic Number               \\
    $Y_{lm}(\theta,\phi)$   &   Spherical Harmonics         \\
    \hline
\end{tabular}
\end{center}
We will use $\mb{k} = (k_x,k_y,k_z)$ to represent the photon wave propagation
vector, and in the case of photoionisation, we will use $\mb{k'} =
(k'_x,k'_y,k'_z)$ for the wave-vector of the ejected electron.
Electronic coordinates with respect to the nucleus will be denoted by 
$\mb{r} = (r,\theta,\phi)$ whereas the ejection angles of a photo-electron will
be denoted by $\Omega = (\Theta,\Phi)$.
In order to avoid confusion with the fine structure constant $\alpha$,
the Dirac alpha matrix will be denoted by $\Alpha$ where
\begin{equation}
    \Alpha = 
    \left(
        \begin{array}{cc}
            0                           &   \mbox{\boldmath $\sigma$} \\
            \mbox{\boldmath $\sigma$}   &   0
        \end{array}
    \right)
\end{equation}
and where $\mbox{\boldmath $\sigma$}$ are the Pauli spin matrices.

When dealing with four component spinors the following definitions have been
adopted.
\begin{itemize}
    \item $\ket{i}$ denotes the $i^{\mathrm{th}}$ state of an atom. For example,
          $\ket{0}$ denotes the ground state, $\ket{1}$ denotes the first
          excited state etc. $\ket{c}$ denotes the continuum or free electron
          state. The state $\ket{i}$ may also be written as $\psi_i$.
    \item $\psi_{ij}$ denotes the $j^{\mathrm{th}}$ component of the 
          $i^{\mathrm{th}}$ atomic state.
    \item $g_i(r)$ and $h_i(r)$ denote the radial components of the $i^{\mathrm{th}}$
          atomic state.
    \item $A_{ij}(\theta,\phi)$ denotes the $j^{\mathrm{th}}$ angular component of the 
          $i^{\mathrm{th}}$ atomic state.

\end{itemize}
\begin{equation}
    \begin{array}{cc}
        \gamma_1 = \sqrt{1 - (\alpha Z)^2} &  \gamma_2 = \sqrt{4 - (\alpha Z)^2}
    \end{array}
\end{equation}
\begin{equation}
    \begin{array}{cccc}
        N_1 = 1 ;& N_2 = \sqrt{2(1 + \gamma_1)} ;& N_3 = 2
        ;&
        \sigma_1 = \left( \frac{2Z}{N_1 a_0} \right)
    \end{array}
\end{equation}
\begin{equation}
    \begin{array}{ccc}
        \epsilon_1 = 
            \left[ 
                1 + \left( \frac{\alpha Z}{\gamma_1} \right)^2
            \right]^{-1/2}
        &
        \epsilon_2 =
            \left[ 
                1 + \left( \frac{\alpha Z}{1+\gamma_1} \right)^2
            \right]^{-1/2}
        &
        \epsilon_3 = 
            \left[ 
                1 + \left( \frac{\alpha Z}{\gamma_2} \right)^2
            \right]^{-1/2}
    \end{array}
\end{equation}


%============================================================================
% SCHRODINGER EQUATION
%============================================================================
\section{The Schr\"odinger Equation and Non Relativistic Wave-functions}
    The Schr\"odinger equation for an electron in a Coulomb potential describes
    the non relativistic structure of a hydrogen atom.
    Ignoring fine structure corrections, the equation can be written as:
    \begin{equation}
        \left[
            -\frac{\hbar^2}{2m} \nabla^2 - \frac{Ze}{r}
        \right] \psi(\mb{r}) = E \psi(\mb{r})
    \end{equation}
    Solutions to this equation~\cite{Bransden-Joachain} give the energy levels 
    of the hydrogen atom as a function of the principal quantum number $n$
    \begin{equation}
        E_n = - \frac{e^2}{4\pi\epsilon_0 a_0} \frac{Z^2}{2n^2}
    \end{equation}
    and wave functions as a function of the three space variables
    $r$, $\theta$ and $\phi$ and the three quantum numbers $n$,$l$ and $m$.
    \begin{equation}
        \psi_{nlm}(r,\theta,\phi) = R_{nl}(r) Y_{lm}(\theta,\phi)
        =
        - \sqrt{
            \left(
                \frac{2Z}{n a_0}
            \right)^3
            \frac{(n - l - 1)!}{2n [(n+l)!]^3}
        }
        e^{-\rho/2} \rho^l L^{2l+1}_{n+l}(\rho) Y_{lm}(\theta,\phi)
    \end{equation}
    where $\rho = 2Zr/na_0$, $a_0 = 4\pi\epsilon_0\hbar^2 / e^2 m$,
    $Y_{lm}(\theta,\phi)$ are the spherical harmonics and 
    $L^{2l+1}_{n+l}(\rho)$ are the associated Laguerre
    polynomials~\cite{Bransden-Joachain}.

    For example, the ground state wave-function of hydrogen 
    ($\psi_{1s}$ or $\psi_{100}$) is defined as:
   	\begin{equation} \label{eq:schrodinger-ground}
		\psi_{1s}(r,\theta,\phi) = \frac{1}{\sqrt{\pi}}
								   \left( \frac{Z}{a_0} \right)^{3/2}
								   e^{-Zr/a_0} .
	\end{equation}

%============================================================================
% DIRAC EQUATION
%============================================================================
\section{The Dirac Equation and Relativistic Wave-functions}
    In order to obtain more accurate energy levels and wave-functions for
    hydrogen, the effects of special relativity which give rise to the fine
    structure in the hydrogen spectrum need to be considered. and as such
    the solutions to the Dirac equation for an electron in a central coulombic
    field need to be used.
    The Dirac equation is shown here (where $\psi$ is a four component spinor
    wave-functions) together with its energy eigenvalue $E_{nj}$
    for hydrogenic atoms.
    \begin{equation} \label{eq:dirac-equation}
    \begin{array}{lcr}
        \left[
            e \Alpha \cdot \mb{p} + \beta mc^2 - \frac{Ze^2}{r}
        \right] \psi = E_{nj} \psi
    &
    \; ; \;
    &
    E_{nj} = \frac{mc^2}{
                \sqrt{
                    1 + \frac{(Z\alpha)^2}{n-j-\frac{1}{2} +
                    \sqrt{(j+\frac{1}{2})^2 - (Z\alpha)^2} }
                }
             }
    \end{array}
    \end{equation}
    Solutions to the Dirac equation for hydrogenic atoms are written in terms of
    four component spinors, separated in radial and angular components as
    defined in~\cite{Bethe-Salpeter}. 
    There are two general spinor solutions $\psi_a$ corresponding to the case
    when $j=l+1/2$ and $\psi_b$ for the case when $j=l-1/2$.
    Note, also that the large and small components of the radial wave-functions are 
    denoted by $g(r)$ and $h(r)$ instead of the more traditional $g(r)$ and $f(r)$, 
    so as to avoid confusion with equation for the form factor which uses the symbol $f$.
    \begin{equation}
    \begin{array}{lr}
        \psi_a = 
        \left(
            \begin{array}{c}
                g(r)    \sqrt{\frac{l+m+1/2}{2l+1}}     Y_{l,m-1/2}(\theta,\phi)    \\
                -g(r)   \sqrt{\frac{l-m+1/2}{2l+1}}     Y_{l,m+1/2}(\theta,\phi)    \\
                -ih(r)  \sqrt{\frac{l-m+3/2}{2l+3}}     Y_{l+1,m-1/2}(\theta,\phi)  \\
                -ih(r)  \sqrt{\frac{l+m+3/2}{2l+3}}     Y_{l+1,m+1/2}(\theta,\phi)
            \end{array}
        \right)
        &
        \psi_b =
        \left(
            \begin{array}{c}
                 g(r)   \sqrt{\frac{l-m+1/2}{2l+1}}     Y_{l,m-1/2}(\theta,\phi)    \\
                g(r)    \sqrt{\frac{l+m+1/2}{2l+1}}     Y_{l,m+1/2}(\theta,\phi)    \\
                -ih(r)  \sqrt{\frac{l+m-1/2}{2l-1}}     Y_{l-1,m-1/2}(\theta,\phi)  \\
                ih(r)   \sqrt{\frac{l-m-1/2}{2l-1}}     Y_{l-1,m+1/2}(\theta,\phi)
            \end{array}
        \right)
    \end{array}
    \end{equation}
    The radial components $g(r)$ and $f(r)$ of the spinors can be written in
    general in terms of the hypergeometric function $F(a,b,x)$. 
    % g(r) radial component
    \begin{equation}
    \begin{split}
    g(r) = - \frac{\sqrt{\Gamma(2\gamma + n' + 1)}}{\Gamma(2\gamma+1)\sqrt{n'!}}
             \sqrt{\frac{1+\epsilon}{4N(N-x)}} 
             \left( \frac{2Z}{Na_0} \right)^{3/2}
             e^{-Zr/a_0}
             \left( \frac{2Zr}{Na_0} \right)^{\gamma-1} 
             \\
             \times
             \left[
                -n' F(-n'+1,2\gamma+1,\frac{2Zr}{Na_0})
                +
                (N-x) F(-n',2\gamma+1,\frac{2Zr}{Na_0})
             \right]
    \end{split}
    \end{equation}
    % h(r) radial component
    \begin{equation}
    \begin{split}
    h(r) = - \frac{\sqrt{\Gamma(2\gamma+n'+1)}}{\Gamma(2\gamma+1)\sqrt{n'!}}
             \sqrt{\frac{1-\epsilon}{4N(N-x)}}
             \left( \frac{2Z}{Na_0} \right)^{3/2}
             e^{-\frac{Zr}{Na_0}}
             \left( \frac{2Zr}{Na_0} \right)^{\gamma-1}
             \\
             \times
             \left[
                n' F(-n'+1,2\gamma+1,\frac{2Zr}{Na_0})
                + (N-x) F(-n',2\gamma+1,\frac{2Zr}{Na_0})
             \right]
    \end{split}
    \end{equation}
    Where we have $x = -(j+1/2)=-(l+1)$ if $j=l+1/2$, $x=j+1/2=+l$ if $j=l-1/2$,
    $\gamma = \pm\sqrt{x^2 - (Z\alpha)^2}$, $\epsilon = E/E_0$ (energy/rest mass
    energy), $n' = \alpha Z \epsilon/\sqrt{1 - \epsilon^2} - \gamma$,
    $N = \sqrt{n^2 - 2n'(k - \sqrt{k^2 - \alpha^2 Z^2})}$, $k = |x|$, and
    $m = \pm(l+1/2)$~\footnote{See Bethe and Salpeter~\cite{Bethe-Salpeter} for a
    detailed explanation of all these constants and definitions}.

    %========================================================================
    % GROUND STATE
    %========================================================================
    \subsection{The Ground State}
    The ground state (or $1S_{\frac{1}{2}}$) of the hydrogen atom corresponds to
    the quantum numbers $n=1$, $l=0$ and $j=\frac{1}{2}$. This corresponds to
    the $j = l+ \frac{1}{2}$ spinor.
    \begin{equation} \label{eq:dirac-ground}
	\ket{1S_{\frac{1}{2}}} = \ket{0} = 
		\left(
			\begin{array}{c}
				A_{01}(\theta,\phi) \; g_0(r) \\
				A_{02}(\theta,\phi) \; g_0(r) \\
				A_{03}(\theta,\phi) \; h_0(r) \\
				A_{04}(\theta,\phi) \; h_0(r)
			\end{array}
		\right)
		=
		\left(
			\begin{array}{c}
				Y_{00} 	g_0(r)	\\
				0	 			\\
				-i \sqrt{\frac{1}{3}} Y_{10} h_0(r) \\
				-i \sqrt{\frac{2}{3}} Y_{11} h_0(r)
			\end{array}
		\right)
    \end{equation}
    %======================================================================
    \begin{equation} \label{eq:dirac-radial}
    \begin{array}{cc}
		g_0(r) = G_0 e^{-\frac{1}{2} \sigma_1 r} r^{\gamma_1 - 1} &
		h_0(r) = H_0 g_0(r) 
    \\
    \\
	    H_0  =  - \sqrt{ \frac{1 - \epsilon_1}{1 + \epsilon_1} }	&
	    G_0  =  \left( \frac{2Z}{a_0} \right)^{3/2} 
		  	   	\sqrt{\frac{1 + \epsilon_1}{2 \Gamma(2\gamma_1 + 1) }}
    \end{array}
    \end{equation}
    %======================================================================
    % FIRST EXCITED STATE
    %======================================================================
    \subsection{The First Excited State}
    \begin{equation}
	\ket{2S_{1/2}} = \ket{1} = 
		\left(
			\begin{array}{c}
				A_{11}(\theta,\phi) g_1(r) \\
				A_{12}(\theta,\phi) g_1(r) \\
				A_{13}(\theta,\phi) h_1(r) \\
				A_{14}(\theta,\phi) h_1(r)
			\end{array}
		\right)
		=
		\left(
			\begin{array}{c}
				Y_{00} 	g_1(r)	\\
				0	 			\\
				-i \sqrt{\frac{1}{3}} Y_{10} h_1(r) \\
				-i \sqrt{\frac{2}{3}} Y_{11} h_1(r)
			\end{array}
		\right)
\end{equation}
\begin{equation}
\begin{array}{cc}
	g_1(r) = e^{-\frac{1}{2} \sigma_2 r} r^{\gamma_1} 
	  		  \left( G'_{1}  \frac{1}{r} - G''_{1} \right) 
    &
   	h_1(r) = H_1 \left( \frac{H'_1 - H''_1 r}{H'''_1 - H''''_1 r} \right) g_1(r)
\end{array}
\end{equation}

\begin{equation*}
\begin{array}{ccc}
	G_1 	 =  \left( \frac{2Z}{N_2 a_0} \right)^{3/2}
		  	  		\sqrt{\frac{2\gamma_1 + 1}{\Gamma(2\gamma_1 + 1)}} 
		  	  		\sqrt{\frac{1 + \epsilon_2}{4N_2 (N_2 + 1)}} 
    \; ; &
	G'_1 	 =  N_2 G_1 \sigma_2^{\gamma_1 - 1} 
    \; ; &
	G''_1 	 =  \left( \frac{N_2 + 1}{2\gamma_1 + 1} \right) G_1 \sigma_2^{\gamma_1}
\end{array}
\end{equation*}
\begin{equation*}
\begin{split}
\begin{array}{ccc}
	H_1 	 =  -\sqrt{\frac{1 - \epsilon_2}{1 + \epsilon_2}}  
    \; ; &
	H'_1 	 =  (2\gamma_1 + 1)(N_2 + 2)  
    \; ; &
	H''_1 	 =  (N_2 + 1)\sigma_2  
\end{array}
\\
\begin{array}{cc}
	H'''_1 	 =  (2\gamma_1 + 1)N_2  
    \; ; &
	H''''_1  =  (N_2 + 1)\sigma_2 
\end{array}
\end{split}
\end{equation*}

    	%=================================================================
		% The Second Excited State
		%=================================================================
    \subsection{The Second Excited State}
    \begin{equation}
	\ket{2P_{1/2}} = \ket{2} = 
		\left(
			\begin{array}{c}
				A_{21}(\theta,\phi) g_2(r) \\
				A_{22}(\theta,\phi) g_2(r) \\
				A_{23}(\theta,\phi) h_2(r) \\
				A_{24}(\theta,\phi) h_2(r)
			\end{array}
		\right)
		=
		\left(
			\begin{array}{c}
				\sqrt{\frac{1}{3}} Y_{10} g_2(r) \\
			    \sqrt{\frac{2}{3}} Y_{11} g_2(r) \\
			    -i Y_{00} h_2(r) \\
				0
			\end{array}
		\right)
\end{equation}

\begin{equation}
\begin{array}{cc}
		g_2(r) = e^{-\frac{1}{2}\sigma_2 r} r^{\gamma_1}
				    \left(
						G'_2 \frac{1}{r} - G''_2
					\right) 
    \; ; &
       	h_2(r) = H_2 
			 \left(
			 	\frac{H'_2 - H''_2 r}{H'''_2 - H''''_2 r}
			 \right) g_2(r)
\end{array}
\end{equation}
\[
    \begin{array}{ccc}
	G_2 	 =  \left( \frac{2Z}{N_2 a_0} \right)^{3/2}
				  \sqrt{\frac{2\gamma_1 + 1}{\Gamma(2\gamma_1 + 1)}} 
				  \sqrt{\frac{1+\epsilon_2}{4N_2 (N_2 - 1)}} 
    \; ; &
	G'_2	 =  (N_2 - 2) \sigma_2^{\gamma_1 - 1} G_2 
    \; ; &
	G''_2 	 =  \left( \frac{N_2 - 1}{2\gamma_1 + 1} \right) \sigma_2^{\gamma_1} G_2
    \end{array}
\]
\[
\begin{array}{cc}
	H_2		 =  -\sqrt{\frac{1-\epsilon_2}{1+\epsilon_2}} 
    &
	H'_2	 =  (2\gamma_1 + 1)N_2 
\end{array}
\]
\[
\begin{array}{ccc}
	H''_2 	 =  (N_2 - 1)\sigma_2 
    &
	H'''_2 	 =  (2\gamma_1 + 1)(N_2 - 2) 
    &
	H''''_2  =  (N_2 - 1)\sigma_2
\end{array}
\]

    	%=================================================================
		% The Third Excited State
		%=================================================================
    \subsection{The Third Excited State}
    \begin{equation}
	\ket{2P_{3/2}} = \ket{3} = 
		\left(
			\begin{array}{c}
				A_{31}(\theta,\phi) g_3(r) \\
				A_{32}(\theta,\phi) g_3(r) \\
				A_{33}(\theta,\phi) h_3(r) \\
				A_{34}(\theta,\phi) h_3(r)
			\end{array}
		\right)
		=
		\left(
			\begin{array}{c}
				\sqrt{\frac{2}{3}} Y_{10} g_3(r) \\
			   -\sqrt{\frac{1}{3}} Y_{11} g_3(r) \\
			 -i \sqrt{\frac{2}{5}} Y_{20} h_3(r) \\
			 -i \sqrt{\frac{3}{5}} Y_{21} h_3(r)
			\end{array}
		\right)
\end{equation}
\begin{equation}
    \begin{array}{cc}
		g_3(r) = G_3 e^{-\frac{1}{2}\sigma_3 r}	r^{\gamma_2 - 1} 
        \; ; &
        h_3(r) = H_3 g_3(r)
    \end{array}
\end{equation}
\[
\begin{array}{cc}
	G_3 =  	\left( \frac{Z}{a_0} \right)^{3/2}
			  \sqrt{ \frac{1 + \epsilon_3}{2\Gamma(2\gamma_2 + 1)} }
			  \sigma_3^{\gamma_2 - 1}
    \; ; &
	H_3  =   -\sqrt{ \frac{1 - \epsilon_3}{1 + \epsilon_3} }
\end{array}
\]

    \subsection{The Free Dirac Electron}
    The continuum state is described by a free Dirac electron state with energy $E$
    and momentum wave-vector $\mb{k'} = (k'_x,k'_y,k'_z)$. 
    Such a state, can be represented by one of the two four-component spinors below.
    \begin{equation}
    \begin{array}{ccccc}
        \ket{\psi_c}_{\uparrow} = 
        \left(
            \begin{array}{c}
                1                   \\
                0                   \\
                \xi k'_z            \\
                \xi (k'_x + ik'_y)
            \end{array}
        \right)
        & ; &
        \ket{\psi_c}_{\downarrow} = 
        \left(
            \begin{array}{c}
                0                   \\
                1                   \\
                \xi(k'_x - ik'_y)   \\
                -\xi k'_z
            \end{array}
        \right)
        & ; &
        \xi = \frac{c\hbar}{E + E_0}
    \end{array}
    \end{equation}
    Where we have $E$ and $E_0$ as the electron's kinetic and rest mass energy
    respectively.
    The $\ket{\psi_c}_{\uparrow}$ represents the electron in the spin up state,
    whereas the $\ket{\psi_c}_{\downarrow}$ represents the electron in the spin
    down state. 















\section{New Results from Relativistic Theory}
    To compute the relativistic version of the normal form factor, we
    need to use the Dirac four component spinors. As for the non
    relativistic case we need to determine the electronic charge
    density given by:
    \(
        \rho(\mb{r}) = \psi^{\dagger} \psi =
            = |\psi_1|^2 + |\psi_2|^2 + |\psi_3|^2 + |\psi_4|^2 
    \)

    Using the ground state solution to the Dirac equation for hydrogenic
    atoms given by equation~\ref{eq:dirac-ground}, the electron density
    is given by:
    \begin{eqnarray*}
        \rho(\mb{r}) &=&  |g_0(r) Y_{00}|^2 + 
                            \frac{1}{3} |i h_0(r) Y_{10}|^2 +
                            \frac{2}{3} |i h_0(r) Y_{11}|^2
       \\
       \rho(\mb{r}) &=&  \frac{1}{4\pi} \left( |g_0(r)|^2 +
                            |h_0(r) \cos \theta|^2  +
                            |h_0(r) \sin \theta \; e^{i\phi}|^2  \right)
       % \\
       % \rho(\mb{r}) &=&  
       = \frac{1}{4\pi} 
                            \left(
                                |g_0(r)|^2 + |h_0(r)|^2
                            \right)
    \end{eqnarray*}

    The radial function $g_0(r)$ and $h_0(r)$ are defined by
    equation~\ref{eq:dirac-radial}. Once again
    using the assumption that a hydrogenic atom is spherically symmetric we
    substitute the electron density $\rho$ into the normal form factor equation
    given by~\ref{eq:nff-spherical}.
    \begin{eqnarray*}
        f_0(q) &=&  \frac{1}{q} 
                    \int_0^\infty \sin (qr) r 
                    \left(
                        |g_0(r)|^2 + |h_0(r)|^2
                    \right) \; dr
                = \frac{2}{q(1 + \epsilon_1)} 
                    \int_0^\infty r |g_0(r)|^2 \sin (qr) \; dr
        \\
                &=& \frac{1}{q \; \Gamma(2\gamma_1 + 1)}
                    \left(
                        \frac{2Z}{a_0}  
                    \right)^{2\gamma_1 + 1}
                    \int_0^\infty r^{2\gamma_1 - 1} e^{-2Zr/a_0} \sin (qr) \; dr
        \\
                &=& \frac{1}{2iq \Gamma(2\gamma_1 + 1)}
                    \left(
                        \frac{2Z}{a_0}
                    \right)^{2\gamma_1 + 1}
                    \left[
                        \int_0^\infty r^{2\gamma_1-1} e^{-(2Z/a_0 - iq)r} \; dr
                    -
                        \int_0^\infty r^{2\gamma_1-1} e^{-(2Z/a_0 + iq)r} \; dr
                    \right]
    \end{eqnarray*}
    Making use of the standard integral
    \(
        \int_0^\infty x^n e^{-ax} \; dx = \frac{\Gamma(n+1)}{a^{n+1}}.
    \)
    and after some algebraic manipulation, we obtain
    equation~\ref{eq:nff-relativistic}.
    Equation~\ref{eq:nff-relativistic} is a \emph{new} result, providing an explicit
    analytic solution for the relativistic normal form factor for hydrogenic
    atoms.
    \begin{equation} \label{eq:nff-relativistic}
        \boxed{
            f_0(q) = 
            \frac{\Gamma(2\gamma_1)}{2iq \Gamma(2\gamma_1+1)}
            \left(
                \frac{2Z}{a_0}
            \right)^{2\gamma_1 + 1}
            \left[
                \frac{
                    \left(
                        \frac{2Z}{a_0} + iq
                    \right)^{2\gamma_1}
                    -
                    \left(
                        \frac{2Z}{a_0} - iq
                    \right)^{2\gamma_1}
                } {
                    \left[
                        \left(
                            \frac{2Z}{a_0} 
                        \right)^2   
                        + q^2
                    \right]^{2\gamma_1}
                }
            \right]
        }
    \end{equation}
    This equation cannot be simplified any further because $\gamma_1$ is not an
    integer. Although the equation has expression containing the complex number
    $i$, when it is evaluated $f_0(q)$ is real valued as expected.

    Previous work in this area include the analytic atomic form factor
    for two electrons in the ground state (K shell) for a helium like atom by
    Bethe and Levinger~\cite{Bethe-Levinger} and Smend and
    Schumacher~\cite{Smend-Schumacher}. The normal form factor for two electrons in the
    K shell as quote in Schaupp et al.~\cite{Schaupp-1983} is
    \begin{equation}
        f_0(q) = \frac{(2Z\alpha)^{2\gamma_1 + 1}}{\gamma_1 q}
                \frac{ \sin[2\gamma_1 \arctan(\frac{q}{2Z\alpha}) ] }{
                    [(2Z\alpha)^2 + q^2]^{\gamma_1}
                }.
    \end{equation}


%-----------------------------------------------------------------------
% THESIS
% AUTHOR    : MICHAEL PAPASIMEON
% EMAIL     : michp@physics.unimelb.edu.au
%-----------------------------------------------------------------------
\documentclass[a4paper,titlepage]{report}

%-----------------------------------------------------------------------
% LATEX PACKAGES
%-----------------------------------------------------------------------
\usepackage{a4wide}
\usepackage{amsmath}
\usepackage{fancybox}
\usepackage{fancyheadings}
\usepackage{float}
\usepackage{tabularx}
\usepackage[dvips]{graphicx}
\usepackage{pslatex}

%----------------------------------------------------------------------
% PAGESTYLE
%----------------------------------------------------------------------

\pagestyle{fancyplain}

%----------------------------------------------------------------------
% LATEX MACROS
%----------------------------------------------------------------------
\restylefloat{table}

\newcommand{\mb}[1]{\mathbf{#1}}
\newcommand{\ket}[1]{|#1 \rangle}
\newcommand{\bra}[1]{\langle #1|}

\setcounter{tocdepth}{10}
\setcounter{secnumdepth}{10}

\setlength{\parindent}{0pt}
\setlength{\parskip}{1.3ex}

\renewcommand{\chaptername}{Section}
\renewcommand{\contentsname}{Table of Contents}
\renewcommand{\bibname}{References}

\newcommand{\HRule}{\rule{\textwidth}{1mm}}
\newcommand{\Trule}{\rule{\textwidth}{0.1pt}}

\newcommand{\maketitlepage}[2]{
    \begin{titlepage}
    \vspace*{\stretch{1}}
    \HRule
    \begin{flushright}
        {\textsf{{ \Huge  #1} }}
        \\[5mm]
        \Trule
        \\[5mm]
        \Large \textsf{#2}
    \end{flushright}
    \HRule
    \vspace*{\stretch{2}}
    \begin{center}
        \Large\textsc{640-497 Advanced Research Project\\
                      Optics Group, 
                      School of Physics\\
                      The University of Melbourne \\
                      \today}
    \end{center}
    \end{titlepage}
}

\newcommand{\Section}[1]{\subsection{#1}}
\newcommand{\Subsection}[1]{\subsubsection{#1}}
\newcommand{\Subsubsection}[1]{\paragraph{#1}}

\newenvironment{Boxedminipage}%
    {\begin{Sbox}\begin{minipage}}%
    {\end{minipage}\end{Sbox}\fbox{\TheSbox}}
\newcommand{\Startbox}[1]{\begin{Boxedminipage}{\textwidth}%
            {\large \sffamily{#1}} \vspace{1mm} \hrule \vspace{2mm} }
\newcommand{\Endbox}{\end{Boxedminipage}}

\newcommand{\bt}{\begin{center}\begin{tabular}%
                {|>{\sffamily}m{6cm}|>{\sffamily}m{2cm}%
                 |>{\sffamily}m{2cm}|>{\sffamily}m{2cm}|} \hline %
                \bfseries{Requirement} & \bfseries{TCSEC} & %
                \bfseries{ITSEC} & \bfseries{CTCPEC} \\ \hline \hline}
\newcommand{\et}{\end{tabular}\end{center}}
\newcommand{\cmp}[4]{#1 & #2 & #3 & #4 \\ \hline}

%----------------------------------------------------------------------
% COMMANDS TO CONSTRUCT FANCY HEADINGS
%----------------------------------------------------------------------

\addtolength{\headwidth}{\marginparsep}
\addtolength{\headwidth}{\marginparwidth}
\setlength{\footrulewidth}{0.4pt}
\setlength{\plainfootrulewidth}{0.4pt}
\setlength{\headrulewidth}{0.4pt}
\setlength{\plainheadrulewidth}{0.4pt}
\setlength{\parindent}{0pt}
\setlength{\parskip}{1.3ex}

\renewcommand{\chaptermark}[1]{\markboth{#1}{}}
\renewcommand{\sectionmark}[1]{\markright{\thesection\ #1}}

\lhead[\fancyplain{\sffamily\thepage}{\sffamily\thepage}]%
      {\fancyplain{\sffamily\rightmark}{\sffamily\rightmark}}
\rhead[\fancyplain{\sffamily\leftmark}{\sffamily\leftmark}]%
      {\fancyplain{\sffamily\thepage}{\sffamily\thepage}}
\cfoot{}

%----------------------------------------------------------------------
% MAIN DOCUMENT
%----------------------------------------------------------------------

\begin{document}
    \maketitlepage{Theoretical Determination of\\
                   \vspace{1mm} 
                   Relativistic Atomic Form Factors\\
                   \vspace{5mm}
                   for Low Z Atoms in the X Ray Regime}%
                   {Michael Papasimeon \\ michp@physics.unimelb.edu.au 
                   \\[5mm]
                   Supervisor: Dr. C.T. Chantler\\
                   chantler@physics.unimelb.edu.au}
    \begin{abstract}
        Form Factor
    \end{abstract}
    \tableofcontents
    \listoftables
    \newpage

%==============================================================
% Introduction
\chapter{Introduction}
%==============================================================
\section{Background}
The use of high energy radiation has many wide spread uses. In
particular, radiation in the X ray regime has applications in medicine,
astronomy and in the study of materials. 
One way to better understand how energetic radiation such as X rays
interact with matter is to consider the most fundamental physical
process involved. In this case, it is the quantum mechanical scattering
of X ray photons off a single atom.
The theoretical calculation of the scattering form factor allows the
determination of a number of quantities such as the scattering cross
section and the material's attenuation coefficient. 

\section{Aims}
The main aim of this project is to compute the form factor for photons
in the X ray regime scattering of small atoms, in particular Hydrogen
and Helium. This is calculated using the relativistic wave-functions as
obtained using the Dirac equation for Hydrogen and a Dirac-Hartree-Fock
numerical approach for Helium. The two components of the form factor
that will be computed are the normal component which has an angular
dependence, and the anomalous component which depends on the energy of
the incident photon.

\section{Atomic Form Factor}
	\subsection{The Physical Model}
	The scattering process considered consists of a single atom in it's
	ground state described by a relativistic four component Dirac spinor
	$\psi$. A photon of momentum $\mb{k}$ and a polarisation $\mb{\hat{e_n}}$
	scatters off the atom. The two processes considered will be:
	\begin{itemize}
		\item Photoionisation of one of the atom's electrons.
		\item Elastic scattering where the momentum of the scattered
			  photon $\mb{k'}$ is the same as the incident photon.
	\end{itemize}

	\subsection{Normal Component}
	The normal component of the atomic form factor shows the angular
	dependence of the scattering process. It can be considered as the
	Fourier transform of the electronic charge distribution and is
	defined as:
	\begin{equation} \label{eq:nff-general}
		f_0(q) = \int e^{i \mb{q} \cdot \mb{r} } \rho(\mb{r}) d\mb{r}
	\end{equation}
	where the momentum transfer is given by $\mb{q} = |\mb{k} - \mb{k'}|$ 
	and $\rho(\mb{r})$ is the electron charge density. In non
	relativistic quantum mechanics, this is defined as 
	$\rho(\mb{r}) = \psi^{*}(\mb{r}) \psi(\mb{r})$ whereas in
	relativistic quantum mechanics it is defined as 
	$\rho(\mb{r}) = \psi^{\dagger}(\mb{r}) \psi(\mb{r})$.
	In the case of a spherically symmetric atom, the normal form factor
	is give by:
	\begin{equation} \label{eq:nff-spherical}
		f_0(q) = 4 \pi \int_{0}^{\infty} \rho(r) \frac{\sin (qr)}{q} r \; dr
	\end{equation}

	\subsection{Anomalous Component}
	The anomalous component of the form factor is defined in terms of
	the scattering amplitude. The cases that are considered are the
	excitement of the ground state electron to a higher energy
	unoccupied state and the excitement of the electron to the
	continuum, that is photoionisation.

	If we consider a relativistic four component spinor $\ket{B}$, which
	is solution to a many particle Dirac equation, the second order
	scattering amplitude (S-matrix) is given by: 
	\begin{equation}
		A = - \sum_P
			\left[
				\frac{\bra{M} O_f^* \ket{P} \bra{P} O_i \ket{N} }
					 {E_N - E_P + \hbar \omega_i + i0_+}
				+
				\frac{\bra{M} O_i \ket{P} \bra{P} O_f^* \ket{N} }
					 {E_n - E_p - \hbar \omega_f - i0_+}
			\right]
	\end{equation}
	where $\ket{N}$ is the initial state, $\ket{M}$ is the final state,
	and $\ket{P}$ are intermediate states. We have the following
	definitions:
	\begin{itemize}
		\item $\hbar \omega_i$ is the energy of the incident photon
		\item $\hbar \omega_f$ is the energy of the scattered photon
		\item $E_N, E_M, E_P$ are the energies of the initial, final 
			  and the intermediate states respectively.
		\item $0_+$ is a small positive number.			  	
	\end{itemize}
	The photon absorption operator $O_{i}$ and the photon emission
	operator $O_{f}^{*}$ are defined as:
	\begin{equation}
		O_i = \sum_j q_j \; \mb{\alpha} \cdot \hat{\epsilon}_i \;
			  e^{i \mb{k_i} \cdot \mb{r_j} }
	\end{equation}
	\begin{equation}
		O_{f}^{*} = \sum_j q_j \; \mb{\alpha} \cdot \hat{\epsilon}_j \;
			e^{i \mb{k_f} \cdot \mb{r_j} }
	\end{equation}
	with $\mb{\alpha}$ is the Dirac alpha matrix operator and
	$\hat{\epsilon}_k$ is the polarisation of the photon.

%==============================================================
% Hydrogen
\chapter{Hydrogenic Atoms}
%==============================================================

\section{Atomic Structure of Hydrogenic Atoms}
	\subsection{Non Relativistic Atomic Structure}
	From Schr\"odinger's equation, we know the ground state of
	hydrogenic atoms is given by:
	\begin{equation} \label{eq:schrodinger-ground}
		\psi_{1s}(r,\theta,\phi) = \frac{1}{\sqrt{\pi}}
								   \left( \frac{Z}{a_0} \right)^{3/2}
								   e^{-Zr/a_0}
	\end{equation}
	where $Z$ is the atomic number and $a_0$ is the Bohr radius.

	\subsection{Relativistic Atomic Structure}
		The relativistic atomic structure of hydrogenic atoms is given
		by the Dirac equation:
		\marginpar{\em Fix Dirac equation}
		\begin{equation}
			%%%%%%%%%%%%%%%%%%%%%%%%%%%%%%%%%%%%%%%%%
			%%%%% FIX THE DIRAC EQUATION FOR HYDROGEN
			%%%%%%%%%%%%%%%%%%%%%%%%%%%%%%%%%%%%%%%%%
			[ e \mb{\alpha} \cdot \mb{p} + \mb{\beta}mc^2 - \frac{Ze^2}{r}] \psi = E\psi
			%%%%%%%%%%%%%%%%%%%%%%%%%%%%%%%%%%%%%%%%%%
		\end{equation}
		where $\psi$ is a four component spinor.
		Solving the Dirac equation for the Hydrogen atom gives the fine
		structure energy levels:
		\begin{equation}
			E_{nj} = mc^2 \left\{
					\left[
						1 + 
						\left(
							\frac{Z \alpha}
								 {n - j - \frac{1}{2} + 
								  \sqrt{(j+\frac{1}{2})^2 - (Z \alpha)^2 } }
						\right)^2
					\right] - 1
					\right\}
		\end{equation}
		We will consider the first four bound states of the hydrogen
		atom and denote them as shown in table~\ref{tbl:hstates}.
\begin{table} 
\begin{center}
\begin{tabularx}{\linewidth}{|X|X|X|X|X|X|}	\hline 
State 	& 				& 					    & $n$ & $l$ & $j$ \\ \hline \hline
Ground	&	$\ket{0}$	& $1s_{\frac{1}{2}}$	& 1   & 0   & $\frac{1}{2}$ \\ 
1st		&	$\ket{1}$ 	& $2s_{\frac{1}{2}}$	& 1   & 0   & $\frac{1}{2}$ \\ 
2nd		&	$\ket{2}$	& $2p_{\frac{1}{2}}$	& 1   & 1   & $\frac{1}{2}$ \\ 
3rd		&	$\ket{3}$	& $2p_{\frac{3}{2}}$	& 1   & 1   & $\frac{3}{2}$ \\ \hline
\end{tabularx}
\caption{First Four States of Hydrogenic Atoms}
\label{tbl:hstates}
\end{center}
\end{table}

	The first four states of the hydrogen atom are listed below in the four
	component spinor formalism. These are the states of the K and L shells of
	the Hydrogen atom. A number of definitions have been made in order
	to simplify the equations. These definitions include:
	\begin{itemize}
		\item $Z$ is the atomic number.
		\item $a_0$ is the Bohr radius.
		\item $\alpha$ is the fine structure constant.
	\end{itemize}
	\begin{eqnarray}
		\gamma_1	& = &	\sqrt{1 - \alpha^2 Z^2}		\\
		\gamma_2	& = &  	\sqrt{4 - \alpha^2 Z^2}		\\
		N_1			& = &	1							\\
		N_2			& = &	\sqrt{2(1+\gamma_1)}		\\
		N_3			& = &	2							\\
		\epsilon_1	& = &	\left[ 
							1 + \left( 
								\frac{\alpha Z}{\gamma_1}
							\right)^2
							\right]^{-1/2}				\\
		\epsilon_2	& = &	\left[
							1 + \left( 
								\frac{\alpha Z}{1 + \gamma_1}
							\right)^2
							\right]^{-1/2}				\\
		\epsilon_3	& = &	\left[
							1 + \left(
								\frac{\alpha Z}{\gamma_2}
								\right)^2
							\right]^{-1/2}				\\	
		\sigma_1	& = &	\left(
								\frac{2Z}{N_1 a_0}	
							\right)
	\end{eqnarray}

		%=================================================================
		% The Ground State
		%=================================================================
		\subsubsection{The Ground State}
\begin{equation} \label{eq:dirac-ground}
	\ket{0} = 
		\left(
			\begin{array}{c}
				A_{01} g_0(r) \\
				A_{02} g_0(r) \\
				A_{03} f_0(r) \\
				A_{04} f_0(r)
			\end{array}
		\right)
		=
		\left(
			\begin{array}{c}
				Y_{00} 	g_0(r)	\\
				0	 			\\
				-i \sqrt{\frac{1}{3}} Y_{10} f_0(r) \\
				-i \sqrt{\frac{2}{3}} Y_{11} f_0(r)
			\end{array}
		\right)
\end{equation}

\begin{equation} \label{eq:dirac-g0}
	\boxed{ 
		g_0(r) = G_0 e^{-\frac{1}{2} \sigma_1 r} r^{\gamma_1 - 1} 
	}
\end{equation}

\begin{equation} \label{eq:dirac-f0}
	\boxed{ 	
		f_0(r) = F_0 g_0(r) 
	}
\end{equation}

\begin{eqnarray*}
	F_0 & = & - \sqrt{ \frac{1 - \epsilon_1}{1 + \epsilon_1} }	\\
	G_0 & = & \left( \frac{2Z}{a_0} \right)^{3/2} 
		  	   	\sqrt{\frac{1 + \epsilon_1}{2 \Gamma(2\gamma_1 + 1) }}
\end{eqnarray*}

		%=================================================================
		% The First Excited State 
		%=================================================================	
		\subsubsection{The First Excited State}
\begin{equation}
	\ket{1} = 
		\left(
			\begin{array}{c}
				A_{11} g_1(r) \\
				A_{12} g_1(r) \\
				A_{13} f_1(r) \\
				A_{14} f_1(r)
			\end{array}
		\right)
		=
		\left(
			\begin{array}{c}
				Y_{00} 	g_1(r)	\\
				0	 			\\
				-i \sqrt{\frac{1}{3}} Y_{10} f_1(r) \\
				-i \sqrt{\frac{2}{3}} Y_{11} f_1(r)
			\end{array}
		\right)
\end{equation}
\begin{equation}
	\boxed{ g_1(r) = e^{-\frac{1}{2} \sigma_2 r} r^{\gamma_1} 
		  		  \left( G'_{1}  \frac{1}{r} - G''_{1} \right) }
\end{equation}

\begin{eqnarray*}
	G_1 	& = & \left( \frac{2Z}{N_2 a_0} \right)^{3/2}
		  	  		\sqrt{\frac{2\gamma_1 + 1}{\Gamma(2\gamma_1 + 1)}} 
		  	  		\sqrt{\frac{1 + \epsilon_2}{4N_2 (N_2 + 1)}} \\
	G'_1 	& = & N_2 G_1 \sigma_2^{\gamma_1 - 1} \\
	G''_1 	& = & \left( \frac{N_2 + 1}{2\gamma_1 + 1} \right) G_1 \sigma_2^{\gamma_1}
\end{eqnarray*}

\begin{equation}
	\boxed{f_1(r) = F_1 \left( \frac{F'_1 - F''_1 r}{F'''_1 - F''''_1 r} \right) g_1(r)}
\end{equation}

\begin{eqnarray*}
	F_1 	& = & -\sqrt{\frac{1 - \epsilon_2}{1 + \epsilon_2}}  \\
	F'_1 	& = & (2\gamma_1 + 1)(N_2 + 2)  \\
	F''_1 	& = & (N_2 + 1)\sigma_2 \\
	F'''_1 	& = & (2\gamma_1 + 1)N_2 \\ 
	F''''_1 & = & (N_2 + 1)\sigma_2
\end{eqnarray*}


		%=================================================================
		% The Second Excited State
		%=================================================================
		\subsubsection{The Second Excited State}
\begin{equation}
	\ket{2} = 
		\left(
			\begin{array}{c}
				A_{21} g_2(r) \\
				A_{22} g_2(r) \\
				A_{23} f_2(r) \\
				A_{24} f_2(r)
			\end{array}
		\right)
		=
		\left(
			\begin{array}{c}
				\sqrt{\frac{1}{3}} Y_{10} g_2(r) \\
			    \sqrt{\frac{2}{3}} Y_{11} g_2(r) \\
			    -i Y_{00} f_2(r) \\
				0
			\end{array}
		\right)
\end{equation}

\begin{equation}
	\boxed{
		g_2(r) = e^{-\frac{1}{2}\sigma_2 r} r^{\gamma_1}
				    \left(
						G'_2 \frac{1}{r} - G''_2
					\right) 
	}
\end{equation}

\begin{eqnarray*}
	G_2 	& = & \left( \frac{2Z}{N_2 a_0} \right)^{3/2}
				  \sqrt{\frac{2\gamma_1 + 1}{\Gamma(2\gamma_1 + 1)}} 
				  \sqrt{\frac{1+\epsilon_2}{4N_2 (N_2 - 1)}} \\
	G'_2	& = & (N_2 - 2) \sigma_2^{\gamma_1 - 1} G_2 \\
	G''_2 	& = & \left( \frac{N_2 - 1}{2\gamma_1 + 1} \right) \sigma_2^{\gamma_1} G_2
\end{eqnarray*}

\begin{equation}
	\boxed{
		f_2(r) = F_2 
				 \left(
				 	\frac{F'_2 - F''_2 r}{F'''_2 - F''''_2 r}
				 \right) g_2(r)
	}
\end{equation}

\begin{eqnarray*}
	F_2		& = & -\sqrt{\frac{1-\epsilon_2}{1+\epsilon_2}} \\
	F'_2	& = & (2\gamma_1 + 1)N_2 \\
	F''_2 	& = & (N_2 - 1)\sigma_2 \\
	F'''_2 	& = & (2\gamma_1 + 1)(N_2 - 2) \\
	F''''_2 & = & (N_2 - 1)\sigma_2
\end{eqnarray*}

		%=================================================================
		% The Third Excited State
		%=================================================================
		\subsubsection{The Third Excited State}
\begin{equation}
	\ket{3} = 
		\left(
			\begin{array}{c}
				A_{31} g_3(r) \\
				A_{32} g_3(r) \\
				A_{33} f_3(r) \\
				A_{34} f_3(r)
			\end{array}
		\right)
		=
		\left(
			\begin{array}{c}
				\sqrt{\frac{2}{3}} Y_{10} g_3(r) \\
			   -\sqrt{\frac{1}{3}} Y_{11} g_3(r) \\
			 -i \sqrt{\frac{2}{5}} Y_{20} f_3(r) \\
			 -i \sqrt{\frac{3}{5}} Y_{21} f_3(r)
			\end{array}
		\right)
\end{equation}

\begin{equation}
	\boxed{
		g_3(r) = G_3 e^{-\frac{1}{2}\sigma_3 r}	r^{\gamma_2 - 1} 
	 }
\end{equation}

\begin{eqnarray*}
	G_3 & = & 	\left( \frac{Z}{a_0} \right)^{3/2}
			  \sqrt{
			  	\frac{1 + \epsilon_3}{2\Gamma(2\gamma_2 + 1)}
			  }
			  \sigma_3^{\gamma_2 - 1}
\end{eqnarray*}

\begin{equation}
	\boxed{
		f_3(r) = F_3 g_3(r)
	}
\end{equation}

\begin{eqnarray*}
	F_3 & = &  -\sqrt{
					\frac{1 - \epsilon_3}{1 + \epsilon_3}
				}
\end{eqnarray*}

	%================================================================
	% HYDROGENIC FORM FACTOR
	%================================================================

\section{Hydrogenic Form Factor -- Normal Component}
	\subsection{Non Relativistic Normal Form Factor}
	Using the non relativistic Schr\"odinger equation, the ground state
	wave-function for a hydrogenic atom is given by
	equation~\ref{eq:schrodinger-ground}.
	The corresponding electron density for the ground state is given by:
	\begin{equation}
		\rho(\mb{r}) = \psi_{1s}^{*} \psi_{1s} 
					 = \frac{1}{\pi a_{0}^{3}} e^{-Zr/a_0}
	\end{equation}
	We know that for a hydrogenic atom we can assume spherical symmetry,
	and therefore by substituting the charge density into
	equation~\ref{eq:nff-spherical} we get the non relativistic normal 
	form factor for hydrogenic atoms.
	\begin{equation*}
		f_0(q) = \frac{4}{q} \left( \frac{Z}{a_0} \right)^3
				 \int_0^\infty e^{-2Zr/a_0} \sin(qr) r \; dr
	\end{equation*}
	\begin{equation}
		\boxed{
			f_0(q) = \left( \frac{2Z}{a_0} \right)^4 
			     	 \left[ \left( \frac{2Z}{a_0} \right)^2 + q^2 \right]^{-2} 
		}
	\end{equation}

	\subsection{Relativistic Normal Form Factor}
	To compute the relativistic version of the normal form factor, we
	need to use the Dirac four component spinors. As for the non
	relativistic case we need to determine the electronic charge
	density.
	\begin{equation*}
		\rho(\mb{r}) = \psi^{\dagger} \psi =
			\left(
				\begin{array}{cccc}
					\psi_1^* & \psi_2^* & \psi_3^* & \psi_4^*	
				\end{array}
			\right)
			\left(
				\begin{array}{c}
					\psi_1 \\
					\psi_2 \\
					\psi_3 \\
					\psi_4
				\end{array}
			\right)
			= |\psi_1|^2 + |\psi_2|^2 + |\psi_3|^2 + |\psi_4|^2
	\end{equation*}
	Using the ground state solution to the Dirac equation for hydrogenic
	atoms given by equation~\ref{eq:dirac-ground}, the electron density
	is given by:
	\begin{eqnarray*}
		\rho(\mb{r}) & = & 	|g_0(r) Y_{00}|^2 + 
							\frac{1}{3} |i f_0(r) Y_{10}|^2 +
							\frac{2}{3} |i f_0(r) Y_{11}|^2
		\\
		\rho(\mb{r}) & = & 	\frac{1}{4\pi} |g_0(r)|^2 +
							\frac{1}{4\pi} |f_0(r) \cos \theta|^2  +
							\frac{1}{4\pi} |f_0(r) \sin \theta \; e^{i\phi}|^2 
		\\
		\rho(\mb{r}) & = & 	\frac{1}{4\pi} 
							\left(
								|g_0(r)|^2 + |f_0(r)|^2
							\right)
	\end{eqnarray*}
	The radial function $g_0(r)$ and $f_0(r)$ are defined by
	equations~\ref{eq:dirac-g0} and~\ref{eq:dirac-f0} respectively. Once again
	using the assumption that a hydrogenic atom is spherically symmetric we
	substitute the electron density $\rho$ into the normal form factor equation
	given by~\ref{eq:nff-spherical}.
	\begin{eqnarray*}
		f_0(q) &=& 	\frac{1}{q} 
					\int_0^\infty \sin (qr) r 
					\left(
						|g_0(r)|^2 + |f_0(r)|^2
					\right) \; dr
		\\
			    &=&	\frac{2}{q(1 + \epsilon_1)} 
					\int_0^\infty r |g_1(r)|^2 \sin (qr) \; dr
		\\
				&=& \frac{1}{q \; \Gamma(2\gamma_1 + 1)}
					\left(
						\frac{2Z}{a_0}	
					\right)^{2\gamma_1 + 1}
					\int_0^\infty r^{2\gamma_1 - 1} e^{-2Zr/a_0} \sin (qr) \; dr
		\\
				&=& \frac{1}{2iq \Gamma(2\gamma_1 + 1)}
					\left(
						\frac{2Z}{a_0}
					\right)^{2\gamma_1 + 1}
					\left[
						\int_0^\infty r^{2\gamma_1-1} e^{-(2Z/a_0 - iq)r} \; dr
					-
						\int_0^\infty r^{2\gamma_1-1} e^{-(2Z/a_0 + iq)r} \; dr
					\right]
	\end{eqnarray*}
	These integrals can be solved because they are of the standard form
	\[
		\int_0^\infty x^n e^{-ax} \; dx = \frac{\Gamma(n+1)}{a^{n+1}}.
	\]
	Solving the integrals and after some algebraic manipulation, the
	relativistic normal form factor for hydrogenic atoms is found to be:
	\begin{equation} \label{eq:nff-relativistic}
		\boxed{
			f_0(q) = 
			\frac{\Gamma(2\gamma_1)}{2iq \Gamma(2\gamma_1+1)}
			\left(
				\frac{2Z}{a_0}
			\right)^{2\gamma_1 + 1}
			\left[
				\frac{
					\left(
						\frac{2Z}{a_0} + iq
					\right)^{2\gamma_1}
					-
					\left(
						\frac{2Z}{a_0} - iq
					\right)^{2\gamma_1}
				} {
					\left[
						\left(
							\frac{2Z}{a_0} 
						\right)^2	
						+ q^2
					\right]^{2\gamma_1}
				}
			\right]
		}
	\end{equation}

	\subsection{Comparing the Non Relativistic and Relativistic Results}
	The relativistic normal form factor in equation~\ref{eq:nff-relativistic}
	can be compared to the non relativistic case by considering a binomial
	expansion of the two terms in the numerator containing complex components.
	\begin{equation*} 
		(a \pm iq)^n \approx a^n \pm ina^{n-1}q + ...
	\end{equation*}
	Using this in equation~\ref{eq:nff-relativistic} we obtain:
	\begin{equation*}
		2 \gamma_1 \frac{\Gamma(2\gamma_1)}{\Gamma(2\gamma_1 + 1)} 
		\left(
			\frac{2Z}{a_0}
		\right)^{4\gamma_1}
			\left[
				\left(
					\frac{2Z}{a_0}
				\right)^2 + q^2
			\right]^{-2\gamma_1}
	\end{equation*}
	In the non relativistic limit we have the case of 
	$\gamma_1 = \sqrt{1 - \alpha^2 Z^2} \approx 1$ and as such the result above
	reduces the normal form factor as obtained using the Schr\"odinger
	wave-functions.
	\begin{equation*}
		f_0(q) \approx
		2 \frac{\Gamma(2)}{\Gamma(3)}
		\left(
			\frac{2 Z}{a_0}
		\right)^4
			\left[
				\left(
					\frac{2 Z}{a_0}
				\right)^2 + q^2
			\right]^{-2}
		=
		\left(
			\frac{2Z}{a_0}
		\right)^4
			\left[
				\left(
					\frac{2Z}{a_0}
				\right)^2 + q^2
			\right]^{-2}
	\end{equation*}

	%===================================================
	% HYDROGENIC FORM FACTOR - ANOMALOUS COMPONENT 
	%===================================================
\section{Hydrogenic Form Factor - Anomalous Component}
	\begin{equation}
		A = -r_0 mc^2 \sum_p^{unocc.} 
			\left[
				\frac{\bra{n} A_f^* \ket{p} \bra{p} A_i \ket{n} }
					 {E_n - E_p + \hbar \omega + i0_+}
				+
				\frac{\bra{n} A_i \ket{p} \bra{p} A_f^* \ket{n} }
					 {E_n - E_p - \hbar \omega - i0_+}
			\right]
	\end{equation}

	%%%%%%%%%%%%%%%%%%%%%%%%%%%%%%%%%%%%%%
	% BOUND-CONTINUUM SCATTERING AMPLITUDE
	%%%%%%%%%%%%%%%%%%%%%%%%%%%%%%%%%%%%%%
	\subsection{Bound-Continuum Scattering Amplitude}

	\subsubsection{Assumptions}
	The case to the bound-continuum scattering amplitude is considered when
	modeling the the ejection of of an electron from the bound state of an atom
	due to an energetic photon. In other words we are looking at the quantum
	mechanical model of photoionisation or the photo-electric effect.

	Here, we will consider the case of a single electron atom in its ground
	state being ionised by an energetic photon, making the assumptions that
	\begin{itemize}
	\item The incident photon is traveling in the $+z$ direction with a
	momentum wave-vector of $\mb{k}$, and it is polarised in the $x$ direction.
	\item The single bound electron is initially in a spin up state, remains in
	a spin up state once ejected and has a rest mass energy of $E_0 = mc^2$.
	\item The electron once ejected has a momentum wave-vector of $\mb{k'}$ and
	has energy $E$ defined by $E^2 = E_0^2 + ( \hbar k' c)^2$, where $k'$ is the
	magnitude of $\mb{k'}$, defined as $k' = \sqrt{k_x^2 + k_y^2 + k_z^2}$.
	\item The ground state of the atom and the free electron will be described
	using the relativistic four component spinor formalism.
	\end{itemize}

	The ground state of the hydrogenic atom is represented by the four spinor in
	equation~\ref{eq:dirac-ground}. The free electron, is represented by the
	continuum state $\psi_c$ specified in the four spinor below:
	\begin{equation} \label{eq:continuum}
		\psi_c = \ket{c} =  e^{i \mb{k'} \cdot \mb{r} }
		\left(
			\begin{array}{c}
				u_1										\\
				u_2										\\
				\xi k'_z u_1 + \xi (k'_x - ik'_y) u_2	\\
				\xi (k'_x + ik'_y)u_1 - \xi k'_z u_2	\\
			\end{array}
		\right)
	\end{equation}
	where we have the following definitions:
	\begin{equation}
		\xi = \frac{c \hbar}{E + E_0}
	\end{equation}
	\begin{equation}
		e^{i \mb{k'} \cdot \mb{r}} =
		e^{k'_x x + k'_y y + k'_z z}
	\end{equation}
	For the case of the ejected electron being in a spin up state:
	\begin{equation}
		u_{\uparrow} = 
		\left(
			\begin{array}{c}
				u_1	\\
				u_2
			\end{array}
		\right)
		=
		\left(
			\begin{array}{c}
				\cos \frac{\theta}{2} 	\\
				\sin \frac{\theta}{2} \; e^{i\phi} 
			\end{array}
		\right)
	\end{equation}

	\subsubsection{The Photo-electric Matrix Element}
	We define the relativistic photo-electric matrix element as follows:
	\begin{eqnarray}
		P_{ab} & = & \left\langle a \left|
						\sum_i e^{\mb{k}\cdot r_i} \alpha_j^{(i)} 
					\right| b \right\rangle \\
		P_{ab} & = & \int a^{\dagger}
						\sum_i e^{\mb{k}\cdot r_i} \alpha_j^{(i)} 
					b  \; d^3\mb{r}
	\end{eqnarray}
	where $\ket{b}$ represents the initial state of the atom, and $\ket{a}$
	represents the ejected electron. For hydrogenic atoms, since we only have
	one electron, the sum is over only this single electron. The polarisation of
	the incident photon is specified by the Dirac alpha matrix in the $j$
	direction, $\alpha_j$. Since we have already assumed that the photon is
	polarised in the $x$ direction then:
	\begin{equation*}
		\alpha_x = 
		\left(
			\begin{array}{cccc}
				0 & 0 & 0 & 1	\\
				0 & 0 & 1 & 0 	\\
				0 & 1 & 0 & 0	\\
				1 & 0 & 0 & 0 	\\
			\end{array}
		\right)
	\end{equation*}
	which implies that the matrix element becomes:
	\begin{equation*}
		P_{ab} = 
		\int
			\bra{a} e^{i\mb{k}\cdot \mb{r}} \alpha_x \ket{b}
		\; d^3 \mb{r}
		=
		\int
			\left(
				\begin{array}{cccc}
					a_1^* &	a_2^* & a_3^* & a_4^*  
				\end{array}
			\right)
			e^{i \mb{k} \cdot \mb{r}}
			\left(
				\begin{array}{cccc}
				0 & 0 & 0 & 1	\\
				0 & 0 & 1 & 0 	\\
				0 & 1 & 0 & 0	\\
				1 & 0 & 0 & 0 	\\
				\end{array}
			\right)
			\left(
				\begin{array}{c}
					b_1 \\
					b_2 \\
					b_3 \\
					b_4
				\end{array}
			\right)
		\; d^3 \mb{r}
	\end{equation*}
	\begin{equation}
		P_{ab} = 
		\int
			a_1^* e^{i \mb{k} \cdot \mb{r}} b_4 + 
			a_2^* e^{i \mb{k} \cdot \mb{r}} b_3 + 
			a_3^* e^{i \mb{k} \cdot \mb{r}} b_2 + 
			a_4^* e^{i \mb{k} \cdot \mb{r}} b_1 
		\; d^3 \mb{r}
	\end{equation}
	After knowing the general form of the matrix element, we can make some
	definitions for some commonly occurring expressions to simplify the
	calculations.
	We first consider the momentum transfer between the photon and the electron.
	We can define a vectorial form as follows:
	\begin{equation}
		\mb{q} = \mb{k} - \mb{k'}
	\end{equation}
	This also implies the momentum transfer in a particular coordinate
	direction, that is $q_i = k_i - k'_i$ where $i = \{x,y,z\}$.
	Making use of spherical polar coordinates ($x = r \sin\phi \cos\theta,
	y = r \sin\phi \sin\theta, z = r \cos\phi$), we can make the following
	definition:
	\begin{equation*}
		\tau(\mb{q},\mb{r}) = e^{i \mb{q} \cdot \mb{r}} 
							= e^{i(q_x x + q_y y + q_z z)} 
							= e^{ir(q_x \sin\phi\cos\theta +
									q_y \sin\phi\sin\theta +
									q_z \cos\phi)}
	\end{equation*}
	From this expression we can make the two useful function definitions:
	\begin{eqnarray}
		\mu = \mu(\mb{q},\theta,\phi) & = & q_x \sin\phi\cos\theta +
								  	  q_y \sin\phi\sin\theta +
								  	  q_z \cos\phi	\\
		\tau = \tau(\mb{q},\mb{r}) & = & e^{ir \mu(\mb{q},\theta,\phi)}
	\end{eqnarray}
	
	Now to solve the matrix element we substitute in for the state $\ket{b}$ the
	ground state of the hydrogen atom as represented by the four spinor in
	equation~\ref{eq:dirac-ground}, and for the state $\ket{a}$ by the continuum
	state represented by equation~\ref{eq:continuum}.
	The second component of the spinor representing the ground state is zero so
	the matrix element becomes:
	\begin{eqnarray*}
		P_{ab} & = & 
		\int
			a_1^* e^{i \mb{k} \cdot \mb{r}} b_4 + 
			a_2^* e^{i \mb{k} \cdot \mb{r}} b_3 + 
			a_4^* e^{i \mb{k} \cdot \mb{r}} b_1 
		\; d^3 \mb{r} \\
		P_{ab} & = & P_1 + P_2 + P_3
	\end{eqnarray*}
	Therefore, the problem is reduced to determining the three integrals, each
	one handled separately, and then combined together at the end.

	\subsubsection{First Component}
	The spinor components for the first integration are:
	\begin{eqnarray*}
		a_1^* & = & e^{-i \mb{k'} \cdot \mb{r} } \cos \frac{\theta}{2} \\
		b_4   & = & -i \sqrt{\frac{2}{3}} f_0(r) Y_{11}
	\end{eqnarray*}
	Substituting these into the integral, with the polar form of the spherical
	harmonic $Y_{11}(\theta,\phi)$ and then making use of the definition of
	$\mu(\mb{q},\theta,\phi)$ we obtain:
	\begin{eqnarray*}
	P_1 & = & i \sqrt{\frac{2}{8 \pi}} \int e^{i (\mb{k} - \mb{k'}) \cdot \mb{r} }
				\cos\frac{1}{2}\theta \sin\theta \, e^{i\phi} f_0(r)
			  \; d^3 \mb{r} \\
	P_1 & = & i \sqrt{\frac{2}{8 \pi}} \int e^{i \mu r}
				\cos\frac{1}{2}\theta \sin\theta \, e^{i\phi} f_0(r)
			    \; d^3 \mb{r} 
	\end{eqnarray*}
	Now converting to polar coordinates $P_1$ becomes:
	\begin{equation*}
	P_1  =  i \sqrt{\frac{2}{8\pi}}
				\int_0^\infty
				\int_0^{\pi}
				\int_0^{2\pi}
					r^2 f_0(r) e^{i \mu r} e^{i\phi} 
					\cos\frac{1}{2}\theta \sin^2 \theta 
				\; dr 
				\; d\theta 
				\; d\phi 
	\end{equation*}
	Ideally, we would like to have an analytic solution for this integral.
	However, if we try to do the angular ($\theta,\phi$) integrals first, we see
	that they are not possible to solve analytically. This is due to the
	fact that $\mu$ is a complicated trigonometric function. The alternative is
	to try to do the radial integral first, which appears that it can be solved
	analytically.
	Considering only the radial component of $P_1$, and substituting for the
	radial function $f_0(r)$ we obtain:
	\begin{eqnarray*}
	P_{1r} & = & - \sigma_1^{\gamma_1 - 1} \left( \frac{2Z}{a_0} \right)^{3/2}
			 \sqrt{ \frac{1-\epsilon_1}{2 \Gamma(2\gamma_1 + 1) } }
			 \int_0^\infty
		 	 e^{i\mu r} e^{-\frac{1}{2} \sigma_1 r} r^{\gamma_1 - 1} r^2 
			 \; dr \\
	P_{1r} & = & -\sigma_1^{\gamma_1 -1 } \left(\frac{2Z}{a_0} \right)^{3/2}
			\sqrt{\frac{1-\epsilon_1}{2\Gamma(2\gamma_1 + 1)}}
			\int_0^\infty 
				e^{-ar} r^{\gamma_1 + 1}
			\; dr
	\end{eqnarray*}
	where $a = \frac{1}{2}\sigma_1 - i\mu$. This is a standard integral 
	($\int_0^\infty e^{-ax} x^n = \Gamma(n+1)/a^{n+1}$), and therefore
	we obtain:
	\begin{equation*}
	P_{1r} = -\sigma_1^{\gamma_1-1} \left( \frac{2Z}{a_0} \right)^{3/2}
			 \sqrt{\frac{1 - \epsilon_1}{2\Gamma(2\gamma_1 + 1)} }
			 \frac{\Gamma(\gamma_1 + 2)}{
			 	\left(
					\frac{1}{2} \sigma_1 - i\mu
				\right)^{\gamma_1 + 2}
			 }
	\end{equation*}
	We can now define a group of constant that will appear in all three
	components.
	\begin{equation*}
	\boxed{
		P_0 = \sigma_1^{\gamma_1 - 1} \Gamma(\gamma_1 + 2)
				\left( \frac{2Z}{a_0} \right)^{3/2}
				\sqrt{ \frac{1 - \epsilon_1}{8 \pi \Gamma(2\gamma_1 + 1)} }
	}
	\end{equation*}
	Now substituting this result back into the equation for $P_1$ together with
	the angular integrals, and after some simplification we obtain:
	\begin{equation}
	\boxed{
		P_1 = -i P_0
		\int_0^{\pi} \int_0^{2\pi}
			\frac{
				e^{i\phi} \cos\left( \frac{\theta}{2} \right) \sin^2(\theta)
			}{
				\left(
					\frac{1}{2}\sigma_1 - i\mu(\mb{q},\theta,\phi)
				\right)^{\gamma_1 + 2}
			}
		\; d\phi \; d\theta
	}
	\end{equation}
	Once again, due to the fact that $\mu(\mb{q},\theta,\phi)$ is a complicated
	trigonometric function, and because $\gamma_1$ is not an integer, the
	angular integrals cannot be solved analytically; however numerical integrals
	are possible.

	\subsubsection{Second Component}
	\begin{equation}
	\boxed{
	P_2 = i P_0
		  \int_0^{\pi} \int_0^{2\pi}
		  	\frac{
				e^{-i \phi} 
				\sin(\theta) \sin\left(\frac{\theta}{2}\right) \cos(\theta)
			}{
				\left(
					\frac{1}{2}\sigma_1 - i\mu( \mb{q},\theta,\phi)
				\right)^{\gamma_1 + 2}
			}
		  \; d\phi \; d\theta
	}
	\end{equation}

	\subsubsection{Third Component}
	\begin{equation}
	\boxed{
	P_3 = \xi P_0
		  \int_0^{\pi} \int_0^{2\pi}
		  	\frac{
				\sin( \theta)
				\left[
					(k'_x - ik'_y) \cos \left( \frac{\theta}{2} \right)
					- k'_z \sin \left( \frac{\theta}{2} \right) e^{-i\phi}
				\right]
			}{
				\left( 
					\frac{1}{2} \sigma_1 - i\mu( \mb{q},\theta,\phi)
				\right)^{\gamma_1 + 2}
			}
		  \; d\phi \; d\theta
	}
	\end{equation}

	\subsubsection{Combined Result}

	\subsection{Bound-Bound Scattering Amplitudes}

%==============================================================
% Helium
\chapter{Helium-like Atoms}
%==============================================================

%==============================================================
% Conclusions
\chapter{Conclusions}
%==============================================================


\end{document}

